\chapter{Tooling \& Utilities}
\label{cha:tooling}

Over the course of this semester we adopted several tools, restructured the code
quite a bit and programmed several utilities to keep the software organized and
automize recurring tasks.

\section{Code Quality}

\subsection{Namespaces}

After the second worksheet there were slightly different versions of several
classes right next to each other and it became hard to keep track of what
belonged together. To better emphasize, which classes are coupled closely
together, we grouped them classes into namespaces. First and foremost all code
now lives under the \texttt{nseof} namespace. Inside of this we divided the code
into

\begin{itemize}
\item \texttt{flowmodels} to group all classes related to flow modelling with
  subnamespaces \texttt{laminar} for the base case, \texttt{turbulent} with
  parent classes for all turbulence models, \texttt{algebraic} for the algebraic
  model from the second worksheet and \texttt{ke} for the $k$-$\epsilon$
  turbulence model
\item \texttt{hdf5} for HDF5 data output
\item \texttt{walldistance} for computing the wall distance of a cell with ANN
\item several more
\end{itemize}

\subsection{Coding Style \& Formatters}

Keeping a consistent coding and formatting style throughout a codebase helps
keeping the code readable and even find bugs. Therefore we first agreed on a
well-defined coding style to adhere to, in our case the one maintained by Google
\footnote{http://google.github.io/styleguide/cppguide.html} because it had the
most familiarity to how you program in Java.

Secondly we configured the C++ code formatter clang-format to adhere to this
style, so that we could automatically reformat the existing codebase and also
check the style of newly written code before committing it into our repository.

\section{Tools}

Most importantly we replaced the original Makefile with CMake
\footnote{https://cmake.org/}. CMake lets you easily script different aspects of
your build. For example we implemented an option to use the special MPI compiler
on the MAC cluster, another one to enable profiling with gprof and a another one
to enable instrumentation with scorep, so that you can configure all those
things command line switches instead of having to edit the Makefile. However the
most important feature for us was that CMake has working dependency checking so
that you can always just make and CMake will figure out, exactly which files
need to be recompiled. This presents a remarkable saving of time compared to the
original Makefile, where this was not guaranteed to work.

Secondly we integrated Google's GTest
\footnote{https://github.com/google/googletest} into our CMake configuration and
created several tests to check the correctness of several of the derivative
functions.

\section{Utilities}

We developed a few utilities to simplify tasks and shrink multi step processes
into a single command.

\subsection{Connecting to the Cluster}

Whenever you want to run a simulation with a lot of cores you need to log on to
the MAC cluster. But this is not so straight forward since the MAC cluster is
only available inside the TUM network. This means you have to tunnel into
university network and then connect throught that. All these things are managed
by the following script, so that you do not have to care.

\bashh{FIGURES/tooling/connect}{1}

Additionally it can also mount your home and scratch directory.

\subsection{Running Jobs}

There is a \texttt{run} script that configures batch jobs and queues them. It is
fully controlled by command line arguments so that you never have to edit
files. The following call runs the scenario \texttt{scenario.xml} with
$8 \times 2$ processes on intel nodes with a maximum run time of 30 minutes,
puts the results into \texttt{./results} and sends you an email, when it is
done.

\bashh{FIGURES/tooling/run}{1}

If you pass \texttt{-l} or \texttt{--local}, you can also use it on your own computer.

\subsection{Performing Scaling Tests}

The \texttt{scale} script performs a weak or strong scaling experiment for
you. For example the following command would test strong scaling with the
scenario \texttt{scenario.xml} on intel nodes with a grid size of
$512 \times 64$ with $1, 4, 8, \dots, 64$ processes and put all results into
\texttt{./results} for later analysis with one of the analysis and plotting
scripts.

\bashh{FIGURES/tooling/scale}{1}

An interesting feature of this script is that it computes the optimal domain
decomposition automatically for all numbers of processes that should be
tested. In the lecture we established that an optimal decomposition consists of
rectangles with a minimal ratio of surface area to volume because that is
actually the same as the ratio of communication to computation.

Say you want to run a $k$-dimensional simulation with $n$ cores on a domain with
$s_{1} \cdot s_{2} \cdots s_{k}$ cells. Then we have to decompose $n$ into
$n_{1}, \dots, n_{k}$ such that $\prod_{i = 1}^{k} n_{k} = n$. This means that
all the $n_{i}$ have to be divisors of $n$ or more specifically all $n_{i}$ have
to be either $1$ or a product of the prime divisors of $n$. Let
$\mathcal{P}_{n}$ be the set of all prime divisors of $n$ with correct
multiplicity. Furthermore let $\mathcal{C}_{n}$ be the set of all ordered
products of $k$-partitions of $\mathcal{P}_{n}$. For example for $k = 3$ we have
$\mathcal{P}_{12} = \{ 2, 2, 3 \}$ and
\begin{equation*}
  \mathcal{C}_{12} = \{ (12, 1, 1), (1, 12, 1), (1, 1, 12), (4, 3, 1), (4, 1, 3), (3, 4, 1), \dots \}
\end{equation*}
Let $(n_{1}, \dots, n_{k}) \in \mathcal{C}_{n}$ be a decomposition. This
decomposition divides the domain into parts of surface area $A$ and volume $V$
where
\begin{equation*}
  A = 2 \cdot \sum_{i = 1}^{k} \prod_{j = 1, j \ne i}^{k} \frac{s_{j}}{n_{j}} \qquad \textit{and} \qquad V = \frac{1}{n} \prod_{i = 1}^{k} s_{i}
\end{equation*}
Now remember that we are trying to minize the ratio of surface area to volume,
i.e.
\begin{align*}
  \frac{A}{V} & = \frac{n}{\prod_{i = 1}^{k} s_{i}} \cdot 2 \cdot \sum_{i = 1}^{k} \prod_{j = 1, j \ne i}^{k} \frac{s_{j}}{n_{j}} = \frac{n}{\prod_{i = 1}^{k} s_{i}} \cdot 2 \cdot \sum_{i = 1}^{k} \frac{\prod_{j = 1, j \ne i}^{k} s_{j}}{\prod_{j = 1, j \ne i}^{k} n_{j}}\\
  & = 2 \cdot \sum_{i = 1}^{k} \frac{\prod_{j = 1, j \ne i}^{k} s_{j}}{\prod_{i = 1}^{k} s_{i}} \cdot \frac{n}{\prod_{j = 1, j \ne i}^{k} n_{j}} = 2 \cdot \sum_{i = 1}^{k} \frac{1}{s_{i}} \cdot n_{i} = 2 \cdot \sum_{i = 1}^{k} \frac{n_{i}}{s_{i}}
\end{align*}
Minimizing this analytically is not trivial since we have to find solutions on a
$k - 1$-dimensional submanifold that is described by a non-linear equation
($\prod_{i = 1}^{k} n_{i} = n$). You can however work out special cases
directly. Let for example be $k = 2$.
\begin{equation*}
  \frac{A}{V} = 2 \cdot \left( \frac{n_{1}}{s_{1}} + \frac{n_{2}}{s_{2}} \right) = 2 \cdot \left( \frac{n_{1}}{s_{1}} + \frac{n}{n_{1}s_{2}} \right) \qquad \left( \frac{A}{V} \right)' = 2 \cdot \left( \frac{1}{s_{1}} - \frac{n}{s_{2} n_{1}^{2}} \right)
\end{equation*}
where the derivative with respect to $n_{1}$ has a root at
$n_{1} = \sqrt{\frac{ns_{1}}{s_{2}}}$. Though you are still not done, because
you have to find the closest integer approximation to
$\left( n_{1}, \frac{n}{n_{1}} \right)$.

That is why the script uses a different algorithm to construct the same integer
solutions directly. It is based on the observation that the subdomains will be
squares when it holds for all $i, j$ that
$\frac{s_{i}}{s_{j}} = \frac{n_{i}}{n_{j}}$, i.e. the number of cores along some
dimensions have the same ration as the number of cells along these dimensions,
because then the subdomains will be $k$-dimensional squares. The integer
solution is constructed iteratively by starting with a decomposition of
$(n_{1}, \dots, n_{k}) = (1, \dots, 1)$ and then multiplying the prime factors
in descending order onto the $n_{i}$ whose ratio of $\frac{n_{i}}{s_{i}}$ is
minimal because that is the one that is the furthest away from its optimal
value, i.e. having the same ratio as all other $n_{i}$. Table
\ref{fig:tooling-decompositions} shows a few example decompositions obtained in
this manner.

\begin{table}[h]
  \centering
  \begin{tabular}{c|cc}
    \hline
    $n$ & $n_{1}$ & $n_{2}$\\
    \hline
    1 & 1 & 1\\
    4 & 4 & 1\\
    8 & 8 & 1\\
    12 & 6 & 2\\
    16 & 8 & 2\\
    20 & 10 & 2\\
    24 & 12 & 2\\
    28 & 14 & 2\\
    32 & 16 & 2\\
    36 & 9 & 4\\
    \hline
  \end{tabular}
  \caption{2-dimensional decompositions of a domain with $512 \times 128$ cells for $n$ processes}
  \label{fig:tooling-decompositions}
\end{table}

Finally the script mangles the cell counts $s_{1}, \dots, s_{k}$ just a little
by rounding up or down to the next multiple of $n_{k}$ so that $n_{k}$ divides
$s_{k}$ because the simulation only accepts decompositions that divide the
domain into parts of exactly equal size.
