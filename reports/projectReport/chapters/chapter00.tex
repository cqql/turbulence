\chapter{Introduction} % (fold)
\label{cha:introduction}

% Peter


Contact phenomena are widespread in mechanical engineering, their appearance ranges from classical technical problems in civil-, automotive- or aeronautical engineering to more recent fields of research as bioengineering. To get an accurate understanding of these phenomena, contact interaction of deformable bodies and related problem domains have been embedded in numerical simulation tools as the Finite Element Method (FEM). To enforce contact constraints in the classical mathematical framework of FEM two common approaches exist, the node-to-segment approach, described e.g. in \citet{laursen1993} or \citet{zavarise2009}, and the segment-to-segment approach. Current developments in computational contact mechanics tend towards so-called mortar methods which enforce contact constraints in a weak sense and thus can be related to the segment-to-segment approach. In contrast to the node-to-segment approach the mortar methods are capable to fulfill simple patch tests and are thus preferred. Patch tests, also referred to as mesh tying or tied contact problems, especially get important when different types of finite element interpolations are used, bodies with different resolutions requirements in particular subdomains are modeled or dissimilar meshes are simulated.

% chapter introduction (end)