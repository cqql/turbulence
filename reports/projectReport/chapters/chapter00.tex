\chapter{Introduction} % (fold)
\label{cha:introduction}

% Peter


The \ke\, turbulence model was implemented as part of the lab course 'Turbulent Flow Simulation on HPC-Systems'. For the purpose of efficiency, stability and testing, a literature study was performed and following aspects have been considered and added to the existing code framework NS-EOF additionally to a basic \ke\, model:

\begin{itemize}
\item wall models
\item new scenarios
\item PETSc-optimization
\item adaptive time stepping
\item restart point \& initialisation of the pressure field in PETSc.
\end{itemize}

\noi Results for a 2D channel flow were compared with literature: the wall near velocities (linear and logarithmic layer) matched well with the expected profile. The terms of the transport equations were predicted qualitatively well due to the chosen wall model (Chien \citep{fan1993}).

\noii Furthermore, comparisons of the DNS results with measurement results were conducted for a backwards facing step scenario and a flow around a cylinder (K\'arm\'an vortex street).
High emphasis was put on the stochastic evaluation. To be able to simulate a flow around a circle, the possibility of importing any arbitrary geometry was implemented.

\noii Extensive thoughts were made how to structure the files of each project in an optimal way (usability \& later clarity).

% chapter introduction (end)