%!TEX root = ./../projectReport.tex

\nomenclature[A]{CPP}{Closest point projection}
\nomenclature[A]{FE}{Finite element}
\nomenclature[A]{FEM}{Finite element method}

\chapter{Turbulence modeling with \ke\, model} % (fold)
\label{cha:turbulence_modeling_with_k_epsilon_model}

This section will give a short introduction to the basic equations solved for in the underlying code: the k-$\varepsilon$ model as a method to model the Reynolds stress tensor will be focused on. To conclude, stability aspects of the model at the wall near region will be mentioned.  

\section{Governing equations} % (fold)
\label{sec:governing_equations}

In general, resolving turbulent structures in time and space in detail is not necessary and also often not possible due to limiting computational resources.
Thus, turbulent behavior of a flow has to be modelled.
Since turbulent behavior is characterized by fluctuations of the flow quantities, they are splitted up as the sum of its mean and the fluctuations around it (Reynolds averaging). This yields exemplarilly for the velocity
\begin{equation}
	u_i(\underline{x},t) = \langle u_i \rangle (\underline{x},t) + u^{\prime}_i(\underline{x},t)
\end{equation}
Therein the operator $\langle \cdot \rangle$ denotes the mean, and the operator $\cdot^{\prime}$ the fluctuation of the respective quantity. Reynolds averaging in the context of incompressible flows is necessary for all velocity components, the pressure and the density.

\section{RANS equations} % (fold)
\label{sec:rans_equations}
Applying this approach to the dimensionless Navier-Stokes-equations and the continuity equation for incompressible flows yields the Reynolds-Averaged Navier-Stokes-equation (RANS):
\begin{equation}\label{eq:conti}
	\abl{\langle u_i \rangle}{x_i} = 0
\end{equation}
and
\begin{equation}
	\label{eq:RANS}
	\abl{\langle u_i \rangle}{t} + \langle u_j \rangle \abl{\langle u_i \rangle}{x_j} = -\frac{1}{\rho} \abl{\langle p \rangle}{x_i}+\frac{1}{Re} \abll{\langle u_i \rangle}{x_j}-\abl{\langle u_i^{\prime}u_j^{\prime} \rangle}{x_j}.
\end{equation}
For the sake of simplicity, the mean operator $\ave{\cdot}$ will not be written in the following since all quantities are averaged. Compared to the Navier-Stokes-equation for laminar flows, on the right hand side of equation \eqref{eq:RANS} the divergence of the Reynolds stress tensor (RST) $\langle -u_i^{\prime}u_j^{\prime} \rangle$ occurs. This term leads to a closure problem and thus has to be modelled.

\noii This can be done by using the physical analogy between molecular and turbulent friction. Dimensional analysis yields for the RST
\begin{equation}
	\label{eq:RST}
	\langle -u_i^{\prime}u_j^{\prime} \rangle = 2\nu_T S_{ij} - \frac{2}{3} k \delta_{ij},
\end{equation}
where $\nu_T$ refers to the turbulent viscosity, $k$ to the turbulent kinetic energy (TKE) and $\delta_{ij}$ to the Kronecker delta.
Inserting equation \eqref{eq:RST} into equation \eqref{eq:RANS} finally yields
\begin{equation}
	\label{eq:RANSmodified}
	\abl{u_i}{t}+\abl{u_i\,u_j}{x_j}=-\frac{1}{\rho}\,\abl{P}{x_i}+2\,\abl{}{x_j}\left( \left(\nu+\nu_T\right)\,S_{ij} \right)+f_i,
\end{equation}
with $f_i$ denoting an additional term for volume forces compared to the RANS equation \eqref{eq:RANS}. Furthermore, it can be shown that
\begin{equation}
	\begin{split}
		\frac{P}{\rho} &= \frac{p}{\rho}+\frac{2}{3}\,k \\
		S_{ij}         &= \frac{1}{2}\left(\abl{u_i}{x_j}+\abl{u_j}{x_i}\right)
	\end{split}
\end{equation}
holds for $P$ and $S_{ij}$. 

\noii Further modelling is necessary to determine the turbulent viscosity $\nu_T$. During the project phase, the approach of $k$-$\varepsilon$ modelling was chosen for that purpose. Therefore, two additional transport equations for the turbulent kinetic energy $k$ and the dissipation $\varepsilon$ have to be solved.
% subsubsection rans_equations (end)

\section{k-$\varepsilon$ transport equations} % (fold)
\label{sec:k_epsilon_transport_equations}

Dimensional analysis yields for the TKE
\begin{equation}
	k = \frac{1}{2} \ave{u_i^{\prime}u_j^{\prime}}
\end{equation}
where $\ave{u_i^{\prime}u_j^{\prime}}$ denotes the trace of the RST. Based on the RANS equation \eqref{eq:RANS} and the continuity equation for incompressible flows, a transport equation for $k$ can be derived.
It can be written as
\begin{equation} \label{tkeTransport}
	\abl{k}{t} + u_i\,\abl{k}{x_i}
	=
	\abl{}{x_i}\left(  B_k \abl{k}{x_i} \right) 
	-
	\varepsilon
	+
	F,
\end{equation}
with
\begin{equation}
	\begin{split}
		B_k &= \nu + \frac{\nu_T}{\sigma_k},\\
		F   &= 2\,\nu_T\,S_{i,j}S_{i,j}.
	\end{split}
\end{equation}
The first term on the left side therein denotes the material derivative of the TKE. Change in TKE can occur by either diffusion (first term on the right side), dissipation (second term on the right side) or production (third term on the right side). As will be seen in the following, the dissipation term can also be generalized to a reaction term.

\noii A transport equation for the dissipation $\varepsilon$, which also occurs in the TKE transport equation \eqref{tkeTransport}, can be derived mathematically. This yields
\begin{equation}
	\begin{split}
		\abl{\varepsilon}{t} + u_i\,\abl{\varepsilon}{x_i}
		&=
		\abl{}{x_i}\left( B_\varepsilon \abl{\varepsilon}{x_i} \right) 
		-
		C_{\varepsilon 2}\,\frac{\varepsilon^2}{k}
		+
		C_{\varepsilon 1}\,F
	\end{split}
\end{equation}
with
\begin{equation}
	\begin{split}
		B_\varepsilon = \nu + \frac{\nu_T}{\sigma_\varepsilon}
	\end{split}
\end{equation}
where $F$ is similarly defined as in \eqref{tkeTransport}. The interpretation of the single terms is analogous to the ones in the TKE-equation. 

\noii The introducing the time scale $T = k/\varepsilon$ (see \citep{adams2015}) yields for the respective transport equation:

\begin{alignat}{3} \label{eq:keTransport00}
		\abl{k}{t} + u_i\,\abl{k}{x_i}
		&=
		\abl{}{x_i}\left(  B_k \abl{k}{x_i} \right) 
		-
		\frac{1}{T}\,k
		&&+
		F, \\ \label{eq:keTransport01}
		\underbrace{\abl{\varepsilon}{t} + u_i\,\abl{\varepsilon}{x_i}}_{1}
		&=
		\underbrace{\abl{}{x_i}\left( B_\varepsilon \abl{\varepsilon}{x_i} \right)}_{2} 
		-
		\underbrace{C_{\varepsilon 2}\,\frac{1}{T}\,\varepsilon}_{3}
		&&+
		\underbrace{C_{\varepsilon 1}\,F}_{4}.
\end{alignat}

\noii Terms $1$, $2$, $3$ and $4$ can be interpeted as the material derivative, dissipation term, reaction term and production term respectively, as mentioned before. For \textit{k}-$\varepsilon$ models the turbulent viscosity can be calculated from the two new variables as
\begin{align}
	\nu_T = C_\mu\,\frac{k^2}{\varepsilon}.
\end{align}


% subsubsection k_epsilon_transport_equations (end)

% subsection governing_equations (end)


\section{Modification 1: Low-Reynolds models for wall near region} % (fold)
\label{sec:modification_1_low_reynolds_models_for_wall_near_region}


When solving the transport equations for $k$ and $\varepsilon$ as presented in equations \eqref{eq:keTransport00} and \eqref{eq:keTransport01}, stability problems at the wall near region can arise. For that reason, near wall models are necessary. They damp several terms of the $\varepsilon$ transport equations. The adapted transport equations can be written as

\begin{alignat}{4} \label{eq:keTransport02}
		\abl{k}{t} + u_i\,\abl{k}{x_i}
		&=
		\abl{}{x_i}\left(  B_k \abl{k}{x_i} \right) 
		-
		\frac{1}{T}\,k
		&&+
		F
		&&&-
		D, \\ \label{eq:keTransport03}
		\underbrace{\abl{\varepsilon}{t} + u_i\,\abl{\varepsilon}{x_i}}_{1}
		&=
		\underbrace{\abl{}{x_i}\left( B_\varepsilon \abl{\varepsilon}{x_i} \right)}_{2} 
		-
		\underbrace{C_{\varepsilon 2}\,f_2\,\frac{1}{T}\,\varepsilon}_{3}
		&&+
		\underbrace{C_{\varepsilon 1}\,f_1\,F}_{4}
		&&&+
		E.
\end{alignat}
The turbulent viscosity is defined as
\begin{align}
\label{eq:nut}
	\nu_T = C_\mu f_\mu \frac{k^2}{\varepsilon}.
\end{align}
Five model constants $C_{\mu}, \sigma_k, \sigma_{\epsilon}, C_{\epsilon 1}, C_{\epsilon 2}$ and five parameters $f_{\mu}, f_1, f_2, D, E$ can be identified in equations \ref{eq:keTransport02}, \ref{eq:keTransport02} and \ref{eq:nut}. The model constants are choosen as proposed in~\citep{fan1993} as
\begin{align}\label{equ:constants}
    C_{\mu} = 0.09 \qquad
    \sigma_k = 1.0 \qquad
    \sigma_{\epsilon} = 1.3 \qquad
    C_{\epsilon 1} = 1.4 \qquad
    C_{\epsilon 2} = 1.8 .	
\end{align}

\noii For the damping parameters, different models have been implemented. In the following, a table with the implemented models and the respective terms for the parameters is shown (see also~\citep{fan1993}).

\begin{adjustbox}{width=\textwidth,totalheight=\textheight,keepaspectratio}
    \begin{tabular}{| >{$}l<{$} | >{$}c<{$} | >{$}c<{$} | >{$}c<{$} | >{$}c<{$} | >{$}c<{$} |}
      \hline
      \text{} & f_{\mu} & f_1 & f_2 & \text{D} & \text{E} \\
      \hline
      \text{Jones and Launder}
      & \exp{\left\lbrack \frac{-2.5}{\left( 1+\frac{R_t}{50} \right)} \right\rbrack}
      & 1.0
      & 1-0.3 \exp{\left(-R_t^2\right)}
      & 2 \nu \left( \abl{\sqrt{k^2}}{y} \right)
      & 2 \nu \nu_T \left( \abll{u}{y} \right)^2 \\
      \text{Chien}
      & 1-\exp{\left( -0.0115 y^+ \right)}
      & 1.0
      & 1-\left( 2/9 \right) \exp{\left\lbrack -\left( \frac{R_t}{6} \right)^2 \right\rbrack}
      & 2 \nu \frac{k}{y^2}
      & -2 \nu \frac{\epsilon}{y^2} \exp{\left( -0.5y^+ \right)} \\
      \text{Lam and Bremhorst}
      & \left\lbrack 1-\exp{\left( -0.0165 R_y \right)} \right\rbrack^2 \left( 1+ \frac{20.5}{R_t} \right)
      & 1+\left(\frac{0.05}{f_{\mu}} \right)^3
      & 1-\exp{\left( -R_t^2 \right)}
      & 0.0
      & 0.0 \\
      \text{Myong and Kasagi}
      & \left\lbrack 1+\frac{3.45}{\sqrt{R_t}} \right\rbrack \times \left\lbrack 1-\exp{\left( -\frac{y^+}{5} \right)} \right\rbrack
      & 1.0
      & \left\lbrack 1-\frac{2}{9} \exp{-\left( \frac{R_t}{6} \right)^2} \right\rbrack \times \left\lbrack 1-\exp{\left( -\frac{y^+}{5} \right)} \right\rbrack
      & 0.0
      & 0.0 \\
      \text{Nagano and Hishida}
      & \left\lbrack 1-\exp{\left( -\frac{y^+}{26.5} \right)} \right\rbrack^2
      & 1.0
      & 1-0.3\exp{\left( -R_t^2 \right)}
      & 2 \nu \left( \abl{\sqrt{k^2}}{y} \right)^2
      & 2 \nu \nu_T \left( 1-f_{\mu} \right) \left( \abll{u}{y} \right)^2 \\
      \text{Fan}
      & 0.4 \frac{f_w}{\sqrt{R_t}}+\left( 1-\frac{f_w}{\sqrt{R_t}} \right) \left\lbrack 1- \exp{\left( -\frac{R_y}{42.63} \right)} \right\rbrack^3
      & 1.0
      & \left\{ 1.0 - \frac{0.4}{1.8} \exp{\left\lbrack -\left( \frac{R_t}{6} \right)^2 \right\rbrack} \right\} f_w^2
      & 0.0
      & 0.0 \\
      \hline
    \end{tabular}
\end{adjustbox}

\noii Therein, for the model of Fan, an additional parameter $f_w$ is necessary. It is defined as
\begin{equation}
f_w = 1 - \exp{\left\{ -\frac{\sqrt{R_y}}{2.30} + \left( \frac{\sqrt{R_y}}{2.30} - \frac{R_y}{8.89} \right) \left\lbrack 1-\exp{\left( -\frac{R_y}{20} \right)} \right\rbrack^3  \right\}}.
\end{equation}
Furthermore, the turbulent Reynolds numbers necessary to determine the factor are defined as
\begin{equation}
	\begin{split}
		R_y = \frac{\sqrt{k\,h}}{\nu} \qquad R_t = \frac{k^2}{\nu\,\varepsilon},	
	\end{split}
\end{equation}
where $h$ denotes the closest wall distance. How to change the chosen wall model can be found in section~\ref{sec:configuration_file}.

\infobox{
nseof::flowmodels::ke::KEStencilF
}

% subsection modification_1_low_reynolds_models_for_wall_near_region (end)

\section{Modification 2: Limitations in transport equations} % (fold)
\label{sec:modification_2_limitation_of_source_and_reaction_terms_in_transport_equtions}


In addition to that, another option to improve stability is implemented. Proposed by \citep{kuzmin2007}, this method aims to avoid negative diffusion and reaction coefficients.


% subsection modification_2_limitation_of_source_and_reaction_terms_in_transport_equtions (end)

\section{Modified equations and discretization} % (fold)
\label{sec:modified_equations_and_discretization}

In addition to the RANS-equations (\ref{eq:conti} and \ref{eq:RANSmodified}), following two equations, which contain both modifications, are solved by NS-EOF:

\begin{alignat}{5} \label{eq:keTransport04}
		\abl{k}{t} + u_i\,\abl{k}{x_i}
		&=
		\abl{}{x_i}\left(  B_k \abl{k}{x_i} \right) 
		-
		f_3\,\frac{1}{T}\,k
		&&+
		F
		&&&-
		D, \\ \label{eq:keTransport05}
		\underbrace{\abl{\varepsilon}{t} + u_i\,\abl{\varepsilon}{x_i}}_{1}
		&=
		\underbrace{\abl{}{x_i}\left( B_\varepsilon \abl{\varepsilon}{x_i} \right)}_{2} 
		-
		\underbrace{C_{\varepsilon 2}\,f_2\,\frac{1}{T}\,\varepsilon}_{3}
		&&+
		\underbrace{C_{\varepsilon 1}\,f_1\,F}_{4}
		&&&+
		E,
\end{alignat}
\noii with
\begin{align}
	f_2&=\left\{\begin{array}{cl} f_2, & \mbox{if }f_2 / T>0 \\ 0, & \mbox{else} \end{array}\right.,
	\\
	f_3&=\left\{\begin{array}{cl} 1, & \mbox{if } 1/T>0 \\ 0, & \mbox{else} \end{array}\right..
\end{align}

\noii For information about implementational aspects of the \textit{k}-$\varepsilon$ model and how it is included in the overall algorithm to solve the flow problem see the next chapter.

% subsection modified_equations_and_discretization (end)

% chapter turbulence_modeling_with_k_epsilon_model (end)

